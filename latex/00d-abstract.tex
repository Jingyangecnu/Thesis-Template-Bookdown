%---------------------------  摘要 --------------------------- 
\chapter*{\markboth{摘要}{摘要}{摘\quad 要}}
\pagenumbering{Roman} 
\medskip
空间广义线性混合效应模型(简称 SGLMM)在现实环境中有广泛的应用,实现其参数估计的算法是研究的重要方面,主要困难是要处理估计中与空间随机效应相关的高维积分,文献中利用似然来处理时,常采用两类方法:一类是蒙特卡罗积分法,另一类是拉普拉斯近似法。这些方法在迭代次数、运算时间或者初始值选取等方面存在不足。为此,我们进行改进:一方面是不同于文献中基于 R 包 geoRglm 实现的 Langevin-Hastings 算法,我们借助 Stan 实现汉密尔顿蒙特卡罗算法(简称 Stan-HMC)去估计 SGLMM 模型的参数,在响应变量服从二项分布和泊松分布的两组模拟实验中,发现 Stan-HMC 算法在保持相似结果下能大大减少迭代次数,还不需要对算法进行调参;另一方面是我们对初始值的选取,在真实数据分析中研究了基于似然函数的参数估计算法,发现这类算法容易陷入局部极值,因此,在小麦数据的分析中借助样本变差图选择初值,在核污染数据的分析中利用剖面似然轮廓来确定合适的初值。


% 文中总结了拉普拉斯近似和蒙特卡罗积分两类计算方法在 SGLMM 模型的参数估计中的应用。论文的创新点其一是借助 Stan 实现汉密尔顿蒙特卡罗算法(简称 HMC)去估计 SGLMM 模型的参数,在响应变量服从二项分布和泊松分布的两组模拟实验中,与基于 R 包 geoRglm 实现的 Langevin-Hastings 算法相比,发现 HMC 算法在保持相似结果下能大大减少迭代次数,还不需要对算法进行调参;其二是在真实数据分析中研究了基于似然函数的参数估计算法,发现这类算法容易陷入局部极值,因此,在小麦数据的分析中借助样本变差图选择初值,在核污染数据的分析中利用剖面似然轮廓来确定合适的初值。

\medskip
\par
{\heiti 关键词 :} 空间随机效应,拉普拉斯近似,蒙特卡罗方法,Stan-HMC

\par
\vspace{1cm}
\noindent\begin{tabular}{l}
\toprule[1pt]\hline
\hspace*{14.5cm}
\end{tabular}

\begin{center}
{\bf \Large Abstract}\\
\vskip 0.6cm
\end{center}
\par

The spatial generalized linear mixed-effects models (SGLMMs) have a wide range of applications in the real world. The high dimensional integral for spatially correlated random effects involved in the parameter estimation is analytically intractable in general. Laplace approximation and Monte Carlo integral, two kinds of calculation methods, are used to estimate parameters of SGLMMs. We use the Hamilton Monte Carlo algorithm (Stan-HMC), programming in Stan language, to estimate the parameters of the SGLMMs. In the simulation experiments in which the response variables draws from the binomial and poisson distribution, respectively. Compared with the Langevin-Hastings algorithm which implemented using geoRglm in R, it is concluded that the Stan-HMC algorithm can greatly reduce the number of iterations while providing really similar results, and does not need to tune the algorithm. We study the parameter estimation algorithms based on likelihood function in real data analysis. It is found that such algorithms are easy to fall into local extremum. Therefore, the variogram is used to choose the initial value in the analysis of wheat data while the profile likelihood contour in nuclear pollution data.

%  main obstacle is to  with the 
%  The algorithm for realizing its parameter estimation is an important aspect of research  
%  
% Effective and efficient algorithms of spatial generalized linear mixed effects models are always pursued by reseachers. The innovation of the thesis has three parts. One is to realize three kinds of parameter estimation methods of spatial generalized linear mixed effects model under the R language programming environment. They are low rank approximation, Monte Carlo Maximum Likelihood and Bayesian Markov Chain Monte Carlo algorithms, respectively. The second is the numerical simulation for comparing the advantages and disadvantages of above algorithms. The third is to implement Bayesian MCMC algorithm based on Stan software, which also applied to analyze malaria data in the Gambia and loa loa data in Cameroon. In conclusion, STAN-MCMC has achieved performance of Bayesian MCMC algorithm almost. Furthermore, due to the high scalability and adaptability of Stan software, the STAN-MCMC algorithm also has these advantages.

\medskip
\par

{\bf Key words:} spatial random effects, laplace approximation, monte carlo methods, Stan-HMC

% 空白页
\newpage 
\mbox{} 

\addtocontents{toc}{\protect\markboth{目录}{目录}} % 设置页眉处目录
